\documentclass[pdftex,12pt,a4paper,english]{article}
\pdfoutput=1 %must be within the first 5 lines according to arXiv

%% character encodings, fonts, language, etc.
\usepackage[T1]{fontenc}
\usepackage{lmodern}
\usepackage[utf8]{inputenc}
\usepackage{babel}

%% typical math resources
\usepackage{amsmath}
\usepackage{amssymb}
\usepackage{amsfonts}

%% for handling multiple authors
\usepackage{authblk}

%% for nice units and fractions
\usepackage{units}

%% for creating hyperlinks in cross-references
\usepackage{hyperref}

%% for paragraphs without indentations
\usepackage{parskip}

%% smaller margins (add [cm] option for even smaller)
\usepackage{fullpage}

%% for, well, fancy headers
\usepackage{fancyhdr}

%% pdf looks better with microtype
\usepackage[kerning,spacing]{microtype}
\microtypecontext{spacing=nonfrench}

%% enable graphics
\usepackage{graphics}
\usepackage[pdftex]{graphicx}

%% for colored text and backgrounds
%\usepackage{color}

\ifpdf
    %% insert license.xmp into the final document
    \usepackage{xmpincl}
    \includexmp{license}

    %% for \pdfdate
    \usepackage{datetime}
    
    \pdfinfo{
        /Author         (Enzymatic Synthesis Group)
        /Title          (Enzymatic DNA Synthesis Roadmap)
        /CreationDate   (D:\pdfdate) %20120221120000
        /ModDate        (D:\pdfdate)
        /Subject        (enzymatic dna synthesis)
        /Keywords       (enzymatic dna synthesis, dna synthesis, dna, polymerase, dna polymerase, protein design, enzyme design, oligonucleotide synthesis)
    }
\fi %pdf

%% just some metadata for \maketitle if it should be used later
%% consider using http://www-spires.fnal.gov/spires/hepnames/authors_xml/
\title{Enzymatic DNA Synthesis Roadmap}
\author{Enzymatic Synthesis Group}

%% for use with authblk
\author[1]{Bryan A. Bishop}
\author[2]{another author}
\author[2]{some other author}

\affil[1]{institute for minimal institutionalization}
\affil[2]{institute for advanced institutionalization}

\begin{document}
    \begin{titlepage}
    %% don't display page numbering or header/footer cruft
    \thispagestyle{empty}

    %% a beautiful title
    \begin{flushleft}
        {   
            \fontsize{60pt}{60pt}
            \selectfont
            {Enzymatic}\\
            {DNA Synthesis}\\
            {Roadmap}
        }   
        \vfill
    \end{flushleft}
\end{titlepage}

%% reset page counter
\setcounter{page}{2}

    \thispagestyle{empty}

%% push everything to the bottom of the page
~\vfill

Copyright \copyright\ 2012 Enzymatic Synthesis Group
\newline
\newline
Creative Commons Attribution-ShareAlike 3.0 Unported
\newline
\newline
CC-BY-SA
\pagebreak

%% reset page counter
\setcounter{page}{3}

    \tableofcontents


    \thispagestyle{plain}
\section{Introduction}
A DNA polymerase is an enzyme that helps catalyze in the polymerization of deoxyribonucleotides into a DNA strand. DNA polymerases are best known for their feedback role in DNA replication, in which the polymerase "reads" an intact DNA strand as a template and uses it to synthesize the new strand. This process copies a piece of DNA. The newly polymerized molecule is complementary to the template strand and identical to the template's original partner strand. DNA polymerases use magnesium ions as cofactors. Human DNA polymerases are 900-1000 amino acids long. \\

aww yeah\\
\ \\
yep..


    
    %\input{tex/bibliography.tex}
    
    \section{Appendix - List of Polymerases}

uniprot ids:\\

B9JWL2
Q0A9G3
B8JAF5
Q2IIA8
B4UDQ9
Q5P4A4
A1KCL7
Q6MQS6
Q7WHW8
Q7W9W1
Q89QU8
A4YRD9
Q57FU7
Q8YEK2
Q8G381
Q62F99
Q63XR8
B8H427
Q9A3J3
Q6NJ04
Q8FRX6
Q6M7C6
Q4JTH1
Q47EP3
Q5HXU1
Q2SJZ6
Q5QUT9
Q605W1
Q7TWL9
Q73U92
O50399
Q5Z1D1
Q6A780
Q48IB1
A6VAU7
Q9I5Q2
Q4KAD4
A4XTN0
Q3KBG7
Q88I82
Q881T7
Q4ZTG0
B2UJ29
Q8XRN1
Q98E34
C3M9Z5
Q7UXK9
Q2IU52
Q07QK9
Q6N8V0
Q139V1
Q21LY9
Q5LVN0
A6UDW8
Q67N73
C5BNE0
A7MUN4
Q87N39
Q8D8I6
Q7MLY4
Q8PN74
Q3BWA6
Q4URY1
B0RY89
Q8PBL8
B2SI43
Q5GX78
Q8UAY9
Q92LA6
Q8UKK2
Q92ZJ6
Q8U642
O28552
A0RYM0
Q5UZ40
Q9HMX8
Q18ER3
Q8THG5
Q46E19
Q12TF2
Q2FSF9
Q59024
Q8TUV3
A2SU54
A4G0G9
A9A8R5
A6VI14
Q8PXH2
A3CXE7
Q6M191
Q2NHG2
O27579
A0B8K5
A6URH2
Q74N29
Q3IU54
Q6L1K9
Q9V2F4
P81409
O57861
Q5JET0
Q9HM33
Q97CR6
Q0W5U6
O28484
B0R7U1
Q9HMR7
Q58113
O27456
Q9V2F3
P81412
O57863
Q9HLK5
Q758V1
Q6FSK8
P0CN25
P0CN24
Q6BQR8
Q6CPH8
O94263
Q6C030
P24482
Q6FXJ8
Q6BX14
Q6CLM5
Q10315
Q6C6M5
P27344
Q750A4
Q6FXD0
Q6BIP4
Q6CJD7
P87174
Q6CHS6
Q04603
P30313
P52028
Q59156
O67779
Q04957
O34996
O51498
O08307
P52027
P00582
P52026
P43741
Q9ZJE9
P56105
P09804
Q9CDS1
O32801
P0A551
P46835
P0A550
Q9HT80
Q9S1G2
Q1RH76
Q92GB7
Q9RAA9
Q9RLB6
O05949
Q9RLA0
Q9F173
P59199
P59200
Q55971
P19821
P80194
O52225
P74933
P21189
P05468
O67125
Q9K838
O34623
O51526
P57332
Q8K9S3
Q89AN8
Q9PPI9
B8GWS6
Q9A700
Q9PJJ7
Q9Z7N8
O84549
Q9RX08
Q8X8X5
Q8FL05
P10443
P43743
Q9ZJF9
P56157
Q9CI70
P63978
Q49405
Q9X7F0
P75404
P63977
Q9JVX8
Q9JXZ2
Q9CPK3
Q9HXZ1
Q9XDH6
Q1RKF9
Q92GB2
Q4UK40
O05974
Q68VX1
P34699
P14567
Q9F1K0
Q5HF71
P63979
P63980
Q6GG04
Q6G8M4
Q8NW58
Q5HNK2
Q8CNX0
Q9Z618
P0C0F3
P0DA74
Q5XBV1
Q8P0S5
P0DA75
P0C0F2
P74750
Q9XDH5
Q9ZHG4
O83675
Q9PQ74
P52022
Q9PGU4
Q87EY0
O68770
P24701
O67725
Q9RCA1
P05649
P33761
P57127
P29439
Q89B36
Q9EVE4
B8GXP6
P0CAU5
Q9PKW4
Q9Z8K0
O84078
P0A990
P0A989
P0A988
P43744
Q9ZLX4
O25242
Q9CJJ1
O54376
P21174
O33914
P24117
P47247
P46387
Q9L7L6
Q50313
Q98RK6
P52851
Q50790
P22838
Q9I7C4
P0A120
P0A121
P31861
Q1RIS7
Q92I37
Q4ULS3
Q9ZDB3
Q68WW0
P26464
P29438
P34029
Q5HJZ4
P0A022
P99103
Q6GKU3
Q6GD88
P0A024
P0A023
Q5HK00
Q8CQK6
P27903
O06672
P59651
P52023
P72856
O83048
Q9KVX5
P52620
O67074
P57337
Q08880
Q89AN3
P03007
P43745
Q9CPE0
O68045
Q1RJM1
Q92GL1
Q4UN31
Q9ZCJ9
Q68W16
P0A1H0
P0A1G9
O83649
P09122
P57553
Q8K983
Q89A95
P06710
P43746
P63976
P47658
P75177
P63975
P74876
Q819Y5
Q9KA72
P13267
Q8XJR3
Q895K2
Q03QS9
Q038M0
Q9CDT7
Q88VK2
A5VJD5
Q1WUF9
Q38W73
Q92C34
Q720A2
Q8Y7G1
A0AIC1
P47277
P75080
P47729
Q8EQU6
Q03FS8
A7X1P4
Q2FHH4
Q2G1Z8
Q2YXK4
Q5HGG7
A6QGG4
P63981
P63982
Q6GHH1
Q6G9U9
P0C1P9
P68852
Q5HPS7
Q8CPG6
Q4L5W6
Q49X49
P63983
P63984
Q8DWE0
P0C0B8
Q04MH6
P0DA76
Q5X9U8
Q8NZB5
Q1J9R4
Q1JJW3
Q1JEV4
Q1J4M0
A2RGI3
Q48R90
Q97SQ2
P0DA77
P0C0B7
Q8DRA5
A4W428
A3CQI2
A4VXT1
Q5M1Y0
Q5M6H0
Q03MX3
Q9ZHF6
A5IJJ8
Q8RA32
Q9PQB4
Q8UFV3
Q9KCU7
P54545
Q8PYH6
P63986
P63985
Q98LV1
Q92QM8
Q8UJK7
Q9K9A8
P54560
Q8PT42
P63988
P63987
Q98JM5
Q92XH8
Q98NW7
Q6FFG4
A3N148
B0BPX9
A0KHD5
B2UKN1
Q0A4X2
A8MGJ5
A7Z6F3
Q81M86
Q818U9
Q635E0
Q64X95
Q8YC76
Q8FW50
Q3ABL2
Q9A5I1
A8AKQ2
Q97MB3
Q18A91
A5N2U5
Q0TQ38
Q8XK37
Q0SSQ2
A3DH61
Q487H6
Q6NGD8
A4QFK8
Q8NNP4
A7MEN4
Q3ZAI3
A5FSS1
Q3ZWB2
B8FBE8
Q6AL48
A7ZHZ2
Q8X7Q1
A7ZWJ6
A1A7U2
Q0TL85
P59477
B1J100
Q47155
Q1RFU0
A4W6W8
Q6D1H8
P58963
B1YK83
B0TZS5
A7NC07
Q2A3L2
B2SH34
A0Q6K9
Q0BM23
Q8REB0
Q5FSM8
B8F542
A1WY69
B0R4P0
Q9HQT4
Q9CE21
A2RNH9
Q02WA3
Q88V07
Q5X7J1
A5IH02
Q5ZY20
Q5WYY8
Q72MY7
Q8F8Q2
Q92A40
C1KWR9
Q71Y46
B8DBV7
Q8Y5T0
A0AK80
Q65TG8
A1U339
A6W1V6
Q8TIW3
A3CSG3
A4YEC1
Q2RJ46
Q5F8N2
Q9JRG1
Q9JYS8
A1KUQ3
A9A3A9
Q8EQ56
B1ZN03
Q6MA55
Q9CNG4
A5D243
Q7N7B6
Q6KZW2
Q6A763
Q48FB2
Q15YP8
A6VA50
Q02I83
Q9I534
Q1I6D5
Q3II35
A4XT24
A5VZT3
Q3KH18
B0KTK6
Q88NK4
Q87Y22
Q4ZWM4
Q8XZ19
Q7USY7
A9MNS1
Q57SU1
Q5PF76
A9MY13
P63990
P63989
A8GAC8
A1S8M3
A3D147
A6WRW5
A9L0T3
Q12Q05
Q086K1
B0TQD1
A3QGY9
Q8EHU9
A8H771
A0KTR5
A8FYV3
Q0HLP0
Q0HY21
A1RMZ5
B1KD33
B2U3S3
Q32J17
Q83M86
Q3Z5A4
A7X420
A6U2Z9
A5IU61
Q2YU32
Q5HEM7
A6QIC3
P63991
P63992
Q6GFG2
Q6G838
P58964
B9DMT3
Q5HN39
Q8CNP3
Q4L7K8
Q49YT9
Q3JZG9
Q8E3I8
Q8DXW9
Q9AK82
A8AUV7
B4SIF5
B2FLR2
Q8DVR7
Q99Y66
Q04M21
P0DA78
Q5XA48
Q8NZI1
Q1JA16
Q1JK63
Q1JF59
Q1J509
A2RCQ1
Q48RJ4
Q97SC7
P0DA79
Q8DQZ7
A4VZ21
A3CKU4
Q5LYC2
Q5M2Y5
Q03J54
Q4JB80
P96022
Q97W02
Q974T8
Q67QM6
P58965
Q73P36
B1AJ79
Q9PQ53
B5ZBT4
A5F5Y1
Q9KPS5
C3LQ59
Q87MB4
Q8DBI7
Q7MID6
Q8PGJ4
Q3BP42
Q4UQI2
Q8PCW8
Q2P7M9
P25615
A1JNY3
A7FLI9
Q1C4D9
Q8ZBZ9
Q1CLD4
A4TPK8
Q66DZ7
O60094
P39985
P28040
P13382
Q9LVN7
P28339
P90829
Q54N97
P54358
P28340
P97283
P52431
Q9LRE6
P30315
O54747
O48901
O48520
P49004
Q19366
Q9W088
P49005
O35654
Q9LRE5
Q6AXY4
P87324
O93610
P46957
P46588
P30316
P15436
Q54RD4
Q07864
Q9WVF7
A7YWS7
Q19196
Q5ZKQ6
Q54Y85
P56282
O54956
Q3SZN5
Q75JQ9
Q9NRF9
Q9JKP7
Q5R4W3
Q642A5
A6QQ14
Q9NR33
Q9CQ36
Q752B8
Q4WXH8
Q6FNY7
P0CN27
P0CN26
Q6BNG2
O93845
Q6CUS7
P87154
Q4PFV5
Q6C4J0
P21951
Q92076
Q27607
P54098
P54099
Q9QYV8
Q91684
Q0VC30
Q9UHN1
Q9QZM2
Q9W6G7
Q9Y767
Q01941
Q12704
P15801
O93745
P95690
O50607
P26811
O93746
Q07635
O05706
P95979
Q9FHA3
Q54SV8
P26019
P09884
O00874
P33609
O48653
Q94636
Q27152
O89042
P27727
Q9DE46
Q27958
Q6DRD3
P06746
Q8K409
P06766
O57383
Q9UGP5
Q4R380
Q9QXE2
Q5RKI3
Q9NP87
Q9JIW4
Q7Z5Q5
Q7TQ07
O75417
P42494
P0C986
P0C984
P0C983
P0C985
Q7T6Y4
O60673
Q61493
A4ZU91
P03261
P87503
P04495
P05664
P06538
P48311
O72539
O72540
Q65946
P87553
Q64751
Q88469
O36363
O29753
P42489
P0C974
P43139
P0C972
P0C971
P0C973
Q64898
Q9T1Q3
Q37882
Q37989
P19894
O64235
Q05254
P03680
P10479
P06950
Q38087
P30314
P06225
P20311
P04415
P19822
P00581
P30319
A7U6F1
Q7SIG7
P03162
P0C691
Q66403
Q1HVC1
P03198
Q3KSP1
P28858
Q6S6P1
P52367
Q9DKT8
P21402
Q6GZR5
Q9E6N9
Q69025
P03161
P03158
P03159
P17100
Q02314
Q91C36
O91533
Q4R1S7
Q4R1R9
P17394
P17393
Q9QAB8
P17395
Q9PX62
Q67925
Q9QBF1
P0C676
Q9E6S5
P0C688
Q9YZR5
P12933
P31870
P03157
Q913A7
Q81165
P0C690
Q69028
P12900
P03155
P24024
P03156
Q9QMI1
P0C679
Q67878
O56655
Q69602
Q80IU7
Q9QAW8
Q80IU4
Q05486
Q8JMY4
Q99HS4
Q99HR5
Q69605
Q8QZQ2
Q9IBI4
P87744
Q9YPV8
Q8JMY7
Q8JN08
Q8JMZ7
Q9J5S2
P08546
P13846
P04293
P07917
P04292
P09854
P07918
P89453
P28857
Q9QJ32
P52342
Q25BI3
P17192
P30028
P17193
A4KX57
P28859
Q196U0
Q9QSK2
Q58295
O27276
P52025
Q5UQR0
P27172
P18131
P41712
Q90162
P30318
Q83948
Q84173
Q6R7C9
P30321
P30320
A7U6F2
A7U6F3
Q6UDK1
P0CL77
P0CL76
P61875
Q9HH06
O59610
P77933
Q51334
P61876
Q85428
O71121
O70736
Q779J8
P24907
O33845
P74918
Q9HH84
P56689
Q9HH05
P30317
Q56366
Q4U3V0
Q9YUS3
Q9YUS2
O57191
P20509
P06856
P33793
P09252
Q4JQU7
P03160
P06275
P12899
P12898
P17396
P11292
O71304
P30322
P22374
P22373
P10582
P33537
P33538
Q01529
Q9P6L6
P14284
P57520
Q8K9B8
Q89AC0
P28630
P43747
P37540
P57435
Q8K9J2
Q89AG8
P28631
P43748
P52024
O69170
P28905
P43749
O68823
P28632
P43750
P0ABT0
P0ABS9
P0ABS8
P0ABT1
Q5XG87
Q6PB75
P63128
Q9Y253
Q9JJN0
Q04049
Q9UNA4
Q6R3M4
P34409
Q9UBT6
Q9QUG2
O74944
P94544
Q7SQ98
P03360
P19560
P25059
P03361
P31792
O93209
P14350
P03362
P14078
P0C211
Q0R5R2
Q4U0X6
P03363
O93215
O12158
O89290
Q9QBY3
Q9QBZ9
O41798
P03369
Q77373
P03366
P04587
Q73368
P03367
P04589
Q75002
P04585
P20875
P0C6F2
Q9QBZ1
P04588
P05961
Q9QBZ5
Q79666
P12497
P18802
P20892
P05959
Q9WC54
Q9WC63
O89940
P24740
Q9Q720
Q9QSR3
P35963
Q9IDV9
O91080
P12499
P18096
P24107
P17757
P15833
Q89928
P18042
Q74120
P05962
P04584
P12451
P20876
Q76634
P04026
P11368
P12894
Q82851
P31623
Q9TTC1
P03355
P03365
P11283
P07572
P03354
Q04095
O92956
P23074
P27401
Q87040
P17283
Q1A249
Q02836
P22382
P05896
Q1A267
P05897
P12502
P19505
Q8AII1
P27973
P27980
P05895
P03364
P04025
P51517
O92815
P03359
Q2F7J3
Q2F7J0
A1Z651
Q07163
O13527
Q09693
Q12490
Q12193
Q12491
P25384
Q03855
Q99231
Q07793
P0C2I2
P0C2I3
Q12472
Q03494
Q07791
Q03612
Q03619
P0CX63
Q12141
Q12269
Q12316
Q12337
P0CX64
Q99315
O13535
P0C2J7
Q7LHG5
P47098
P47100
P47024
Q12088
P0C2I5
P0C2I6
P0C2I7
P0C2J3
P0C2J5
Q04711
Q03434
Q04214
Q04670
Q12112
Q99337
Q12273
Q92393
Q12113
Q12501
O31902
Q12414
P0C2I9
P0C2J0
P0C2J1


\end{document}
